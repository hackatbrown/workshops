\newcommand{\mytitle}{Vim Basics} 
\newcommand{\myauthor}{Alex and Eli}

\documentclass{article}
\usepackage{graphicx,datetime,fancyhdr,amsmath,amssymb,amsthm,subfig,url,hyperref}
\usepackage{enumitem,titling,minted,float,multicol}
\usepackage[margin=1in]{geometry}
\setlistdepth{9}
\setlist[itemize,1]{label=$\bullet$}
\setlist[itemize,2]{label=$\circ$}
\setlist[itemize,3]{label=$-$}
\setlist[itemize,4]{label=$*$}
\setlist[itemize,5]{label=$\to$}
\setlist[itemize,6]{label=$\mapsto$}
\setlist[itemize,7]{label=$\bullet$}
\setlist[itemize,8]{label=$\bullet$}
\setlist[itemize,9]{label=$\bullet$}
\renewlist{itemize}{itemize}{9}
%----------------------- Macros and Definitions --------------------------

\setlength{\parindent}{0pt}
\renewcommand{\theenumi}{\bf \Alph{enumi}}
\fancypagestyle{plain}{}
\pagestyle{fancy}
\fancyhf{}
\fancyhead[LO,RE]{\sffamily\bfseries\large \mytitle}
\fancyhead[RO,LE]{\sffamily\bfseries\large \myauthor}
\fancyfoot[RO,LE]{\sffamily\bfseries\thepage}
\renewcommand{\headrulewidth}{1pt}
\renewcommand{\footrulewidth}{1pt}
\graphicspath{{figures/}}
\setlength{\droptitle}{-1cm}
%-------------------------------- Title ----------------------------------

\title{Vim Basics}

\author{Alex and Eli}

\begin{document}
\subsection*{Agenda for Presentation}
\begin{itemize}
\item Start with the demo, and why vim:
\begin{itemize}
\item Just in the terminal, makes it easy and efficient to edit code without
		using a mouse (though you cam use a mouse with vim)
\item Faster to type, refactor, see errors, change parts of code
\item Large community, lots of plugins, lots of helpful stack-overflow-like posts
		you can find with a quick google search.
\item Not required.
\item see also: emacs. We wont talk about this, but emacs is another cool editor
		with a steep learning curve and lots of community and plugins.
\item Demo itself:
\begin{itemize}
\item Syntastic, with some OCaml or C code.
\begin{itemize}
\item This one is especially useful, and we can show you how to download and
				install it at the end.
\end{itemize}
\item splitting
\item conque-term
\item autocomplete (tab, control-n, control-p)
\item nerd-tree
\end{itemize}
\end{itemize}
\item Insert Mode
\begin{itemize}
\item How to use \verb|i|, etc.
\item Introduce the general concept of ``mode,'' or at least the word
\end{itemize}
\item Normal Mode:
\begin{itemize}
\item Essential commands
\begin{itemize}
\item Saving
\item Quitting
\end{itemize}
\end{itemize}
\item This is enough to do most of the basic editing operations most are accustomed
	to with other editors.
\begin{itemize}
\item Include some basic source code examples
\item Moreover, it can run normally in the terminal!
\item But, it is no more powerful than other text editors like sublime text that
		someone might be using.
\end{itemize}
\item Copying and Pasting: scaling up the commands, starting to talk about
	composition.
\begin{itemize}
\item \verb|y|, \verb|d| and \verb|p|. Copying, cutting, and pasting.
\begin{itemize}
\item Example here is two large blocks of text, where you have to cut and paste
			one below the other.
\end{itemize}
\item Visual Mode
\begin{itemize}
\item It turns out, most traditional text editors come prepackaged with a
			visual mode. For example, if you open up sublime, here is how you enter
			``visual mode:'' (holds down mouse button, highlights some text).
\item Use this to highlight the entire paragraph, cut it, and paste it below.
\end{itemize}
\item Now, introduce the idea of composing operations with numbers and other
		operations, by changing \verb|vjjjjjjjlllllllllldjjjjjp| to \verb|d7j5jp|
\end{itemize}
\item Cool things you can do from normal mode:
\begin{itemize}
\item \verb|w| and \verb|e| for navigating. Show how you can compose these with
		numbers, and \verb|d|.
\item This is how most of these modifiers in normal mode work. There are so many
		things that we aren't showing you, but they follow more or less the same
		principles.
\item Search with \verb|/|. The file can have the words ``delete me'' on line
		1000 and they can search for it.
\end{itemize}
\item Practice the same basic exercises, but this time using the new commands we've
	introduced.
\item Setting up the \verb|.vimrc|.
\begin{itemize}
\item defaults in their home directory should be fine
\end{itemize}
\item Pathogen:
\begin{itemize}
\item install syntastic: 2.2.1--2.2.2 in \url{https://github.com/scrooloose/syntastic}
\item more practice
\end{itemize}
\item Conclusion:
\begin{itemize}
\item Composition/modes really are a great foundation for thinking about text
		editing.
\item The best way to improve is to concentrate on one or two features that you
		want to learn at a time and use those a lot until they're second nature.
\item Provide a list of plugins folks might enjoy.
\item Things you may be interested in:
\begin{itemize}
\item Tabs
\item Marks
\item Registers
\item Folds
\item Different movement commands
\item Reading from shell
\item And So much more!\end{itemize}
\end{itemize}
\end{itemize}

\end{document}
